%-------------------------------------
% Portada
%-------------------------------------
\newpage
\input{portada_cap.tex}
\begin{center}
\vspace*{15em}
{\huge\bfseries\color{blue}{Capítulo 2}\par}
{\huge\bfseries\color{blue}{El Abordaje del Problema}\par}
\end{center}
%------------------------------------------
%Capítulo 2
%------------------------------------------
\chapter{Capítulo 2}
%----------------------------
% Subtítulo
%----------------------------
\section{Límites y alcances}
%----------------------------
% Contenido
Debido a que se pretende medir la eficiencia de los algoritmos de Aprendizaje Automático, se utilizarán las siguientes métricas:
Exactitud.
Tasas de error.
Precisión.
F1-Score.
%----------------------------
\section{Justificación}
%----------------------------
% Contenido
%----------------------------
\section{Limpieza y procesamiento de la base de datos}
%----------------------------
% Contenido
%----------------------------
\section{Análisis exploratorio de los datos}
%----------------------------
% Contenido
%----------------------------
\section{Ajuste de los parámetros de los modelos}
%----------------------------
% Contenido
Entre los algoritmos que se han utilizado para la detección de spam por email están:
Filtrado Bayesiano.
SVM (Support Vector Machine).
Clasificador kNN.
Red Neuronal.
Clasificador AdaBoost.
Algunos enfoques novedosos que se han propuesto para este fin:
Aprendizaje Profundo.
Redes Generativas Antagónicas.

%----------------------------