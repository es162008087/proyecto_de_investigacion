\thispagestyle{empty}
\chapter*{Introducción}
\addcontentsline{toc}{chapter}{Introducción}
\pagenumbering{arabic}
%-----------------------------
% Texto de introducción
%-----------------------------
% \lipsum[2-3] %Ejemplo de texto
El llamado  spam es un tipo de mensajes de texto no deseado ni solicitado, que es recibido a través de alguna plataforma digital, principalmente en el correo electrónico, pero también  puede llegar en mensajes SMS (Short Message Service) o en plataformas sociodigitales como Twitter, etc.

Debido a que este tipo de mensajes es de una inmensa variedad y a que puede saturar los servidores de correo electrónico, existe un gran cantidad de investigación y herramientas computacionales dedicados a su filtrado.  Por ejemplo los filtros Bayesianos los cuales buscan ocurrencias de palabras particulares en los mensajes. Así, para una palabra particular \textit{w}, la probabilidad de que aparezca un  spam es calculada por el número de veces que aparece en un conjunto grande de correos de  spam y el número de veces que aparece en un conjunto grande de correos no spam.

Este trabajo se enfoca en los mensajes de  spam que llegan por email. Mediante el uso de técnicas de Aprendizaje de Máquina, también llamado Aprendizaje Computacional o Aprendizaje Automático se podría facilitar la correcta clasificación de mensajes, para detectar y detener los mensajes de spam. Dicho Aprendizaje Automático, puede ser Supervisado o No Supervisado.

El objetivo, en el Aprendizaje Supervisado, es que se aprenda una función que se aproxime mejor a los resultados deseados. Se tiene un conocimiento previo de cuáles deberían ser los valores de salida para las muestras. El Aprendizaje Supervisado se realiza normalmente en el contexto de la clasificación.
En el Aprendizaje No Supervisado, el objetivo es inferir la estructura natural presente  dentro de un conjunto de puntos de datos. Algunos casos de uso comunes son el análisis exploratorio y la reducción de la dimensionalidad. El Aprendizaje No Supervisado se realiza normalmente en el contexto de la agrupación.


Algoritmos de Aprendizaje Supervisado

Árboles de decisión: El objetivo es crear un modelo que prediga el valor de una variable objetivo en función de varias variables de entrada.

Clasificación de Naïve Bayes: un modelo probabilístico que se utiliza para tareas de clasificación, basado en el teorema de Bayes.

Regresión por mínimos cuadrados: Utilizada para tratar de encontrar el límite de decisión óptimo
Regresión Logística: Probabilidad de que la salida del modelo, entre 0 y 1, pertenezca a un determinado grupo o clase.

SVM: Máquinas de vectores de soporte (Support vector machine), separa las clases en dos espacios lo más amplio posibles, mediante un hiperplano de separación mediante un vector de soporte.
Random forest,: consiste en una gran cantidad de árboles de decisión individuales que operan como un conjunto.

Redes neuronales: Simulando el funcionamiento de las redes neuronales, es posible utilizar algoritmos que combinan varios de los metodos mencionados, para hacer más eficiente la clasificación.


Algoritmos de Aprendizaje No Supervisado

Algoritmos de clustering: Trata de encontrar una estructura de agrupamiento en una colección de datos no etiquetados.

Análisis de componentes principales: Utiliza una transformación ortogonal para convertir un conjunto de observaciones de variables en un conjunto de valores de variables linealmente no correlacionadas llamadas componentes principales.



